\documentclass{webofc}
\usepackage[varg]{txfonts}   % Web of Conferences font
\usepackage{url}
\begin{document}
\title{System Performance and Cost Modelling in LHC computing}
\author{\firstname{Catherine} \lastname{Biscarat}\inst{1} \and
        \firstname{Tommaso} \lastname{Boccali}\inst{2} \and
        \firstname{Daniele} \lastname{Bonacorsi}\inst{3} \and
        \firstname{Concezio} \lastname{Bozzi}\inst{4,5} \and
        \firstname{Davide} \lastname{Costanzo}\inst{6} \and
        \firstname{Dirk} \lastname{Duellmann}\inst{4} \and
        \firstname{Johannes} \lastname{Elmsheuser}\inst{7} \and
        \firstname{Eric} \lastname{Fede}\inst{8} \and
        \firstname{José} \lastname{Flix Molina}\inst{9} \and
        \firstname{Domenico} \lastname{Giordano}\inst{4} \and
        \firstname{Costin} \lastname{Grigoras}\inst{4} \and
        \firstname{Jan} \lastname{Iven}\inst{4} \and
        \firstname{Michel} \lastname{Jouvin}\inst{10} \and
        \firstname{Yves} \lastname{Kemp}\inst{11} \and
        \firstname{David} \lastname{Lange}\inst{12} \and
        \firstname{Helge} \lastname{Meinhard}\inst{4} \and
        \firstname{Michele} \lastname{Michelotto}\inst{13} \and
        \firstname{Gareth Douglas} \lastname{Roy}\inst{14} \and
        \firstname{Andrew} \lastname{Sansum}\inst{15} \and
        \firstname{Andrea} \lastname{Sartirana}\inst{16} \and
        \firstname{Markus} \lastname{Schulz}\inst{4} \and
        \firstname{Andrea} \lastname{Sciab\`a}\inst{4} \and
        \firstname{Oxana} \lastname{Smirnova}\inst{17} \and
        \firstname{Graeme} \lastname{Stewart}\inst{4} \and
        \firstname{Andrea} \lastname{Valassi}\inst{4} \and
        \firstname{Renaud} \lastname{Vernet}\inst{8} \and
        \firstname{Torre} \lastname{Wenaus}\inst{7} \and
        \firstname{Frank} \lastname{Wuerthwein}\inst{18}
}

\institute{Univ. Grenoble Alpes, CNRS, Grenoble INP, LPSC-IN2P3, Grenoble, France\and
           INFN Sezione di Pisa, Pisa, Italy \and
           INFN Sezione di Bologna, Universit\`a di Bologna, Bologna, Italy \and
           European Organisation for Nuclear Research (CERN), Geneva, Switzerland \and
           Universit\`a e INFN, Ferrara, Ferrara, Italy \and
           Department of Physics and Astronomy, University of Sheffield, Sheffield, United Kingdom \and
           Physics Department, Brookhaven National Laboratory, Upton, NY, USA \and
           Centre de Calcul de l'IN2P3 du CNRS, Lyon, France \and
%           Laboratoire d'Annecy de Physique des Particules (LAPP) and CNRS/IN2P3, Annecy, France \and
           Centro de Investigaciones Energ\'eticas Medioambientales y Tecnol\'ogicas (CIEMAT), Madrid, Spain \and
           LAL, Universit\'e Paris-Sud and CNRS/IN2P3, Orsay, France \and
           Deutsches Elektronen-Synchrotron, Hamburg, Germany \and
           Princeton University, Princeton, NJ, USA \and
           INFN Sezione di Padova, Universit\`a di Padova, Padova, Italy \and
           SUPA - School of Physics and Astronomy, University of Glasgow, Glasgow, United Kingdom \and
           STFC Rutherford Appleton Laboratory, Didcot, United Kingdom \and
           Laboratoire Leprince-Ringuet, Ecole Polytechnique, CNRS/IN2P3, Universit\'e Paris-Saclay, Palaiseau, France \and
           Lunds Universitet, Fysiska Institutionen, Avdelningen f\"or Experimentell H\"ogenergifysik, Box 118, 221 00 Lund, Sweden \and
           University of California, San Diego, La Jolla, CA, USA
          }

\abstract{The increase in the scale of LHC computing expected for Run
  3 and even more so for Run 4 (HL-LHC) over the next ten years will
  certainly require radical changes to the computing models and the
  data processing of the LHC experiments. Translating the requirements
  of the physics programmes into computing resource needs is a
  complicated process and subject to significant uncertainties. For
  this reason, WLCG has established a working group to develop
  methodologies and tools intended to characterise the LHC workloads,
  better understand their interaction with the computing
  infrastructure, calculate their cost in terms of resources and
  expenditure and assist experiments, sites and the WLCG project in
  the evaluation of their future choices.  This working group started
  in November 2017 and has about 30 active participants representing
  experiments and sites. In this contribution we expose the
  activities, the results achieved and the future directions.}

\maketitle

%Sciaba
\section{Introduction}
The computing infrastructure for the LHC experiments, managed via the
WLCG project~\cite{wlcg}, has been successfully operating since 2008,
with a good match between the resources needed by the experiments and
those made available by the funding bodies. In the last few years
though it became increasingly clear that Run 3 (for ALICE and LHCb)
and Run 4, or HL-LHC (for ATLAS and CMS) will determine very
significant increases in the scale of computing, which will not be
simply accommodated by technological improvements in the likely
scenario of a flat budget evolution. Simply extrapolating the current
software performance to the expected trigger rates and average number of overlapping events (pile-up)
produces a $O(10)$ discrepancy between the needed and the affordable
levels of computing and storage capacity. For this reason,
``revolutionary'' changes in the software and the computing models of
the experiments will be absolutely necessary.

The need of a joint working group between WLCG and the
HSF~\cite{hsf} dedicated to the study of performance and cost of
computing became apparent and it finally started at the end of
2017, with a long term roadmap that extends to the start of
HL-LHC. About thirty members, from experiments, sites and IT and software
experts participate in the group activities. The initial focus has
been on improving the understanding of current workloads and to
establish methodologies and tools to analyse their performance; however,
some thought has already been given to the exploration of future
scenarios and to try to quantify the gains that could be achieved
through paradigm changes. Currently the most important areas of work are:
\begin{itemize}
\item the collection of reference workloads from each experiment, to conduct
  performance studies in controlled and repeateable conditions;
\item the definition of the metrics that best characterise the applications and the implementation of tools to measure them;
\item the adoption of a common framework for estimating resource needs;
\item the adoption of a common process to evaluate the cost of an infrastructure
  as a function of the experiment needs;
\item preliminary studies to explore possible savings in different areas.
\end{itemize}



%Markus
\section{Workload characterisation and metrics}
\begin{figure}[h]
  \centering
  \includegraphics[height=4cm]{CHEP-Model-2.png}
  \hfill
  \includegraphics[height=4cm]{prmon.png}
  \caption{{\em (left)} Workload resource requirements depend on the
    pile-up and the trigger rate. The model predicts the demands on the
    different resources. The Fabric Mapper abstracts the
    characteristics of a fabric and produces estimates for
    throughputs. Optional sites can add their local cost structure to
    the model. {\em (right)} Memory consumption over the execution of
    an ATLAS Digi-Reco job from PrMon; the different processing steps
    are clearly visible. The periods with very low memory consumption
    are due to the merging of intermediate files}
  \label{fig:mapping}
\end{figure}

To model the behaviour of the workloads of the LHC experiments it is
essential to understand how the different capabilities that a site
provides impact their throughput. To limit the complexity of a
performance model, it is desirable to minimise the number of
performance characteristics; this requires to find a set of
``orthogonal'' metrics to which the throughput is sensitive. This
knowledge can then be used by software developers to avoid bottlenecks
by balancing resource usage, and by site managers to optimise their
expenditure. The balance between the amount of memory per core, the
disk performance and the memory speed and latency can be optimised on
the basis of the characterisation of the relevant
workloads. Figure~\ref{fig:mapping} (left) illustrates this relationship.

Finding such a metric set is far from trivial, given the complexity of
workloads and their production environments. We started from a
representative collection of reference workloads for each experiment.
Then, we listed several metrics, grouped in various categories (CPU,
I/O, memory, storage, swap and network) and assessed their
relevance. Finally, we focused on those that can be measured without
significant interference with the running workload on nodes. A tool,
PrMon~\cite{prmon}, originally developed by ATLAS, has been
generalised and allows to measure a large set of parameters, using
information from the operating
system. Figure~\ref{fig:mapping}~(right) shows, as an example,
measurements of the memory usage during the execution of a workload
that performs digitisation with pile-up events and reconstruction for
Monte Carlo events.  PrMon allows to estimate which capabilities of a
system limit the throughput and how efficiently multi-process and
multi-threaded applications use their allocated cores. To study the
dependencies on latency, bandwidth and memory restrictions, another
tool has been developed that runs workloads repeatedly with
increasingly restricted bandwidth, memory and increased latency,
measuring the throughput for each configuration.

Yet another tool, Trident~\cite{trident}, was developed to provide a
convenient access to the information contained in CPU hardware
counters.  These counters provide information on the level of
parallelism exploited by a workload, memory access, cache utilisation
and vector capabilities, etc.~in an abstract and CPU-model dependent
way.  This gives the developer an estimate on how much improvement is
at best possible in exploiting the available resources and
quantitatively understand the limiting factors. The site managers can
use the insights gained from Trident in decisions concerning
trade-offs between different capabilities, like memory speed vs. size
or system disk speed and network.

\subsection{Next steps}
While the time series measured by PrMon give valuable insights, they
cannot be directly used as input for modelling the behaviour of the
workloads. Work on this process has started to parametrise metric time
series, as processing can be described as sequence of steps, each one
looping over events. During each of these steps the resource usage can
be described by a small number of parameters and the number of
processed events. This, in combination with the dependencies on
latency, bandwidth and memory, will allow to predict the throughput of
the current workloads under different conditions.

For the resource modelling under future running conditions the
dependency on the average pile-up has to be integrated into the
model. For data size and simulation time, the dependency is at most
linear. For reconstruction, an extrapolation to HL-LHC levels is
currently very uncertain and it would be exponential with the current
algorithms. ATLAS plans to address this issue by introducing closely
spaced tracking layers, to provide precise starting vectors, while CMS
will take advantage of a very high time resolution to distinguish
between hits from different collisions.


%Sartirana
\section{Resource estimation}
In order to sensibly plan their activity and estimate their costs,
experiments need to translate the requirements of their physics
programmes into computing resources needs. They thus need a modelling
tool that takes as input the details of the physics activities and the
features of the computing model and outputs the amount
and characteristics
of computing resources required.

Currently, all LHC experiments have their own tools for this task,
often some complex spreadsheets. Our purpose is to provide a common
framework that could standardise, generalise and possibly improve what
each experiment does, in its own specific way, today. This framework
should be as generic and customisable as possible, it should define a
schema for the parameters that describe the expected activity and the
characteristics of the computing model, and provide some standard
calculations on these parameters.  Besides providing the experiment
with a standardised way to compute their needs, such framework will
allow us to easily play with different computing model scenarios and
explore potential gains.

\subsection{Current status}
A first version of the framework \cite{ourresmodel} was obtained by
forking, refactoring and slightly generalising existing code used by
CMS to estimate HL-LHC requirements~\cite{cmsresmodel}. It is a set of
modules and scripts that read some input files and produces plots and
tables with the required storage and CPU resources. Inputs are loaded
hierarchically so that we can easily overwrite a generic set of
definitions with different special scenarios. This is handled by the
classes and functions defined in the {\it ResourceModel}
module. Parameters include:
\begin{itemize}
\item LHC parameters: trigger rates, live fractions, shutdown years, etc.
\item Computing model: event sizes and processing times, software improvement factors, etc.
\item Storage model: numbers of versions, replicas, etc.
\item Infrastructure: capacity model, Tier-1 disk and tape, etc.
\end{itemize}
The framework computes the number of data and Monte Carlo events per
year using the LHC parameters and some physics information like the
required Monte Carlo/data ratio. The functions defined in the {\it
  CPUModel} and {\it StorageModel} modules translate the number of
events into CPU and Storage requirements for the different activities
(reconstruction, Monte Carlo, analysis).  The module {\it ModelOut}
defines functions to create plots and tables.

\subsection{Next steps}
This first version of the framework elicited strong interest from
other LHC experiments and has been tentatively agreed as a common
basis for future development. Generalising it to all LHC experiments
will require rewriting some of the most CMS-specific parts and making
the parameters schema more flexible. Even if a common framework was
not finally adopted by all LHC experiments, having them to adopt a
similar approach will greatly benefit the ability to produce
consistent and flexible resource estimates.  Other foreseen
improvements are a generic time granularity and an estimation of
network resources.


%Renaud
\section{Site cost estimation}
The purpose of site cost estimation is to understand and measure what
the data centres typical expenses are, and predict what they may be in
the future. Our approach is to model those expenses taking into
account the diversity of national contexts across sites in terms of
funding, procurement procedures, and local market conditions.  These
results will help experiments plan new computing strategies that will
improve the cost-effectiveness of their resource usage.

\subsection{First results}
Some of the first elements to address are the diversity of expenses
across sites and their definition.  A simple and quick exercise was
made by four sites, aiming to get a first estimate of the financial
cost to run a given workflow and to store a given amount of data, in
terms of IT resources and power consumption.  The answers to this
exercise differed up to a factor of two from site to site.
The reasons were found to originate from the intrinsic variety of
costs, the measurement method, and the understanding of what a given
metric means.

Those results clearly showed the need of a consolidated and common
model and method to measure costs across WLCG data centres. For
example, one may consider the cost of a server providing a given
capacity, including (or not) the equipment that is shipped with it
(rack, switch, adaptor etc.).  One may also include (or not) in a unit
of capacity of tape storage the investments made in library, drives,
disk cache etc.\ that are needed for such system to work.  Finally,
the measurement of the electrical consumption may include (or not) the
Power Usage Effectiveness of the data centre in order to take into
account UPS or HVAC system contributions to the final power bill.  It
is therefore fundamental for site cost estimation to establish a
precise definition of the cost-related metrics in order to build a
reliable model.

\subsection{Next steps}
An attempt to address the total cost of ownership (TCO)
through cost modeling was shown in~\cite{costmodel}. This model
assumes that a data centre invests each year a constant budget in the
following assets: batch system, disk storage and tape system
capacities. If this hypothesis is satisfied (even roughly), one can
show that budget ($B$) and available capacity ($K$) over time are
bound together by a quantity ($c^*$) that depends on hardware cost
evolution and lifetime:

\begin{equation}
    B (t) = K (t) \times c^* (t)
    \label{eq:costmodel}
\end{equation}

In the case where hardware unitary costs decrease exponentially over
time with a rate $r$ (e.g. 0.2 for a 20\% yearly decrease) and
hardware is replaced after $\tau$ years, one can show that

\begin{equation}
c^*(t)=c(t)\frac{r}{1-(1-r)^\tau}
\end{equation}
and the site capacity for a flat budget turns out to be exponentially
increasing.

This model however does not address all the components of a TCO, like
manpower.  Site cost studies may leverage this model to estimate the
financial impact of a variation in the usage that experiments make of
data centre resources.  But the quantity $c^*$ may differ
significantly from site to site, so an extension of this model at
global scale will have to take into account as many as site-dependent
parameters as possible to establish $c^*$.

In order to get a better idea of the variations of the expenditure at
different sites, a survey is being conducted at all Tier-1 sites,
although the results are not yet available.


%Markus
\section{Areas of improvement}
Since the start of the LHC the community constantly improved the
throughput of the main workflows; however, studies conducted by the
Understanding Performance team and this working group found several
areas in which potential improvements could be
achieved.

\subsection{Compiler and software improvements}
WLCG sites provide a variety of CPU models and HEP code is usually
built to run on the oldest available architectures. This, and
advancements in compilers motivated further studies.

Studies on GEANT simulation, reconstruction and NLO generator code
show gains by compiler and link-based optimisations to be around
20-25\%, including compilation for individual target CPUs, the use of
Intel’s commercial compiler and the use of feedback-directed
optimisation (AutoFDO from Google and Intel). Compiler-based
vectorisation of current production code showed little or no
improvement. Reducing the overhead of shared libraries by building
large libraries resulted in large gains on older architectures (e.g.,
a 10\% on Ivy Bridge for ATLAS simulation). It has to be noted that
the use of feedback-directed optimisation requires that the objects
are built statically, which is not always trivial.

From code profiling and a detailed analysis of the dynamic use of
memory allocation, it is known that the current code spends up to 25\%
of the time on memory/object management, due to the frequent creation
and destruction of small objects~\cite{fomtools}. A 60-90\% of
allocations exist for less than $100\mu$s and are smaller than 64
bytes.  By using better strategies for object management, such as
object pools and static vectors, this could be reduced to less than
10\% with relatively minor code refactorisation. At the same time the
data structure layout can be improved for more efficient utilisation
of caches and improved memory access.

The current HEP code executes on modern cores only $0.8-1.5$
instructions per cycle (IPC). With the large vector registers provided by
current CPUs the theoretical limit is above 20 IPC for fully
vectorised code, and tuned complex code for HPC systems can reach
about 4 IPC, which can be seen as an upper limit for our code
base. Approaching this limit requires at least a change of the used
data structures and a re-implementation or factorisation of our
algorithms, which should be taken into consideration when
designing new algorithms.

Gains from new algorithms and the impact of new detector components
cannot be treated here. As the development of the cellular automaton
based, HLT track reconstruction code for the ALICE HLT has shown,
large factors $O(100)$ can in some cases be achieved~\cite{rohr}.

\subsection{Storage}
The LHC community has recently agreed to consider a scenario where
managed storage is consolidated at a few, very large data centres (the
``data lake'') as the most promising to achieve significant cost
savings~\cite{cwp}.

For what concerns operational effort, the 2015 WLCG site
survey~\cite{survey} showed that on average Tier-1 sites require 2.5
full time equivalent (FTE) for operating storage and \mbox{Tier-2} sites 0.75 FTE, with a weak
dependence on the amount of storage. By concentrating managed storage
at a few sites and using only disk caches at
most sites, one can estimate a decrease of the overall number of FTE
for storage operations from around 100 to around 60.

Nearly $80\%$ of used disk in WLCG consists of data formats used for
analysis. Data popularity studies show that on average datasets exist
at $\sim2$ sites and are accessed less than 10 times over a period of
six months, with most accesses in the first month.  We can argue that
data needs to be stored only once in the data lake, and temporarily
cached as needed depending on the client load. Less popular data may
be purged from disk and kept on tape.
Again, analysis of popularity data and storage system monitoring
indicates that only a small fraction of the produced data is active at
any time and a significant fraction (15-20\%) of the data could be
moved to a different storage layer.

Potential savings depend on the retention strategy and the use of
hierarchical storage: what is gained from replacing some amount of
disk with tape is partially offset by the need of more tape drives and
some added complexity in the data migration.

The concentration of data at a few sites requires limiting the impact
of bandwidth limitations and latency on the workload throughput.  We
measured the impact of latency on throughput of various workloads,
showing that for latencies up to 25ms the reduction of throughput can
be limited to 5\% when using Xcache~\cite{xcache} as an additional layer
(table~\ref{tab:latency}).
\begin{table}
  \centering
  \caption{ Summary of data access studies for different types of
    workloads, data location with respect to the processing node,
    corresponding network latency, access method and relative running
    time with respect to processing local data}
  \label{tab:latency}
  \begin{tabular}{llrlr}
    \hline
    Workload & Data location & Latency (ms) & Access method & Rel. time \\\hline
%    ATLAS Digi-Reco & on node & 0 & direct & 1.00 \\ 
    ATLAS Digi-Reco & remote & ~25 & direct & 1.90 \\ 
    ATLAS Digi-Reco & rem., empty cache & ~25 & cached & 1.08 \\ 
%    ATLAS Digi-Reco & rem., pop. cache & ~25 & cached & 1.04 \\ 
%    ATLAS Derivation & on node & 0 & direct & 1.00 \\ 
%    ATLAS Derivation & close storage & < 1 & direct  & 1.02 \\ 
    ATLAS Derivation & remote & ~25 & direct & 8.30 \\ 
    ATLAS Derivation & rem., empty cache & ~25 & cached & 1.05 \\ 
%    ATLAS Derivation & rem., pop. cache & ~25 & cached & 1.03 \\
%    CMS Digi-Reco & local & 0 & added latency & 1.00 \\
%    CMS Digi-Reco & remote, added latency & 5 & direct & 1.01 \\
    CMS Digi-Reco & remote, added latency& 10 & direct & 1.04 \\
    CMS Digi-Reco & remote, added latency& 20 & direct & 1.11 \\
    CMS Digi-Reco & remote, added latency& 50 & direct & 1.24 \\\hline
  \end{tabular}
\end{table}

Data redundancy within storage systems is achieved either by
replication (more performant but expensive) or some form of error
encoding (cheaper but less performant). Even larger savings could be
achieved by not using data redundancy at all and re-staging from tape,
or even regenerating, any lost data.  Based on the observed disk
failure rate of about 1\% per year in the CERN disk storage, the
relative total cost of storage and computing at CERN (around 4
HS06/TB)\footnote{HEPSpec06 is the benchmark commonly used to measure
  CPU power for HEP applications~\cite{hs06}.}, the amount of CPU time
to generate AOD events ($\sim 850$ HS06$\cdot s$) and their size
($\sim 400$ kB), one can estimate that the computing cost to
re-generate the AOD data lost to disk failures is $\sim 20\%$ of the
cost to make the storage redundant by full replication.

In reality, most of the times the lost data would be replicated
elsewhere; one can conservatively estimate that 30-50\% of the disk
costs can be saved in this way. For this approach to be efficient, the
process of recreating individual files has to be automated, which is
highly desirable since other failure modes (software bugs, human
errors, etc.) lead to data loss on a comparable scale.

\subsection{Gains from improvements in operations}
Scheduling inefficiencies in WLCG can arise from a mismatch between
cores in a system and memory requirements, a mismatch between
requested cores and cores grouped on nodes (the tessellation problem),
batch system or pilot service inefficiencies, delays due to data
staging, I/O waits etc. Site managers have identified some of these
problems and $20-30\%$ of resources may be lost due to them. With
advanced backfilling and more complex job placement strategies,
efficiencies above 90\% can be reached, at the expense of added
complexity in workload management systems. Another source of
scheduling inefficiencies stems from breaking the processing chains
from raw data to data analysis objects into several individual steps
that exchange data via files: especially when parallel processing
threads/processes write individual files, the merging steps create
inefficiencies, as they are single threaded. By using asynchronous I/O
these inefficiencies can be reduced. The overall impact is difficult
to measure, but from ATLAS step chain activity logs it can be
estimated to be about 5\%.

Losses due to occasional job failures are unavoidable. Currently most
of the steps can recover from failure with the help of automated retry
mechanisms, but complete jobs are sometimes run up to 20 times before
successful completion. The overall loss in walltime due to job
failures is between 10 and 15\%. Improved procedures to make jobs more
resilient or fail very early can reduce these losses.
Table~\ref{tab:pgain} summarises the identified potential gains.

\begin{table}
  \centering
  \caption{Estimated potential gains and associate efforts}
  \label{tab:pgain}
  \begin{tabular}{llll}
    \hline
    Change & Effort for sites & Effort for users & Potential gain \\\hline
    Managed storage only at few sites & some on large sites & little & $-40\%$ effort \\
    Reduced data redundancy & some on large sites & some & $-30$-$50\%$ cost \\
    Scheduling and site inefficiencies & some & some & $+10$-$20\%$ CPU  \\
    Reduced job failure rates & little & some - massive & $+5$-$10\%$ CPU \\
    Compiler and build improvements & none &  little - some & $+15$-$20\%$ CPU \\
    Improved memory usage & none & some &  $+10$-$15\%$ CPU \\ 
    Exploiting modern CPU arch. &  none & massive & $+100\%$ CPU \\\hline
  \end{tabular}
\end{table}


%Sciaba
\input{Conclusions}

\urlstyle{same}
\begin{thebibliography}{}
\bibitem{wlcg}
Worldwide LHC Computing Grid, \url{http://wlcg.web.cern.ch}
\bibitem{hsf}
HEP Software Foundation, \url{https://hepsoftwarefoundation.org/}
\bibitem{prmon}
G.~Stewart, A.S.~Mete, \url{https://doi.org/10.5281/zenodo.2554202}
\bibitem{trident}
S.~Muralidharan, D.~Smith, \textit{Trident: A three pronged approach to analysing node utilisation}, these proceedings
\bibitem{ourresmodel}
A.~Sartirana, \url{https://github.com/sartiran/resource-modeling/tree/wlcg-wg-new}
\bibitem{cmsresmodel}
D.~Lange {\em et al}, \textit{Computing Resources: Meeting the demands of the high-luminosity LHC physics program}, these proceedings
\bibitem{costmodel}
R.~Vernet, J.\ Phys.: Conf.\ Ser.\ \textbf{664}, 052040 (2015)
\bibitem{fomtools}
S.~Kama and N.~Rauschmayr, J.\ Phys.: Conf.\ Ser.\ {\bf 898} 072031 (2017)
\bibitem{rohr}
D.~Rohr {\em et al}, J.\ Phys.: Conf.\ Ser.\ \textbf{396} 012044 (2012)
\bibitem{cwp}
HEP Software Foundation, \textit{A Roadmap for HEP Software and Computing R\&D for the 2020s}, arXiv:1712.06982
\bibitem{survey}
M. Alandes Pradillo {\em et al}, J.\ Phys.: Conf.\ Ser.\ \textbf{664} 032025 (2015)
\bibitem{xcache}
W.~Yang {\em et al}, \textit{Xcache in ATLAS Distributed Computing}, these proceedings
\bibitem{hs06}
M.~Michelotto {\em et al}, J.\ Phys.: Conf.\ Ser.\ \textbf{219} 052009 (2010)
\bibitem{hepixbm}
D.~Giordano, M.~Alef and M.~Michelotto, \textit{Next Generation of HEP CPU Benchmarks}, these proceedings
\end{thebibliography}


\end{document}
