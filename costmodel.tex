\documentclass{webofc}
\usepackage[varg]{txfonts}   % Web of Conferences font
\usepackage{url}
\begin{document}
\title{System Performance and Cost Modelling in LHC computing}
\author{\firstname{Catherine} \lastname{Biscarat}\inst{1} \and
        \firstname{Tommaso} \lastname{Boccali}\inst{2} \and
        \firstname{Daniele} \lastname{Bonacorsi}\inst{3} \and
        \firstname{Concezio} \lastname{Bozzi}\inst{4,5} \and
        \firstname{Davide} \lastname{Costanzo}\inst{6} \and
        \firstname{Dirk} \lastname{Duellmann}\inst{4} \and
        \firstname{Johannes} \lastname{Elmsheuser}\inst{7} \and
        \firstname{Eric} \lastname{Fede}\inst{8} \and
        \firstname{José} \lastname{Flix Molina}\inst{9} \and
        \firstname{Domenico} \lastname{Giordano}\inst{4} \and
        \firstname{Costin} \lastname{Grigoras}\inst{4} \and
        \firstname{Jan} \lastname{Iven}\inst{4} \and
        \firstname{Michel} \lastname{Jouvin}\inst{10} \and
        \firstname{Yves} \lastname{Kemp}\inst{11} \and
        \firstname{David} \lastname{Lange}\inst{12} \and
        \firstname{Helge} \lastname{Meinhard}\inst{4} \and
        \firstname{Michele} \lastname{Michelotto}\inst{13} \and
        \firstname{Gareth Douglas} \lastname{Roy}\inst{14} \and
        \firstname{Andrew} \lastname{Sansum}\inst{15} \and
        \firstname{Andrea} \lastname{Sartirana}\inst{16} \and
        \firstname{Markus} \lastname{Schulz}\inst{4} \and
        \firstname{Andrea} \lastname{Sciab\`a}\inst{4} \and
        \firstname{Oxana} \lastname{Smirnova}\inst{17} \and
        \firstname{Graeme} \lastname{Stewart}\inst{4} \and
        \firstname{Andrea} \lastname{Valassi}\inst{4} \and
        \firstname{Renaud} \lastname{Vernet}\inst{8} \and
        \firstname{Torre} \lastname{Wenaus}\inst{7} \and
        \firstname{Frank} \lastname{Wuerthwein}\inst{18}
}

\institute{Univ. Grenoble Alpes, CNRS, Grenoble INP, LPSC-IN2P3, Grenoble, France\and
           INFN Sezione di Pisa, Pisa, Italy \and
           INFN Sezione di Bologna, Universit\`a di Bologna, Bologna, Italy \and
           European Organisation for Nuclear Research (CERN), Geneva, Switzerland \and
           Universit\`a e INFN, Ferrara, Ferrara, Italy \and
           Department of Physics and Astronomy, University of Sheffield, Sheffield, United Kingdom \and
           Physics Department, Brookhaven National Laboratory, Upton, NY, USA \and
           Centre de Calcul de l'IN2P3 du CNRS, Lyon, France \and
%           Laboratoire d'Annecy de Physique des Particules (LAPP) and CNRS/IN2P3, Annecy, France \and
           Centro de Investigaciones Energ\'eticas Medioambientales y Tecnol\'ogicas (CIEMAT), Madrid, Spain \and
           LAL, Universit\'e Paris-Sud and CNRS/IN2P3, Orsay, France \and
           Deutsches Elektronen-Synchrotron, Hamburg, Germany \and
           Princeton University, Princeton, NJ, USA \and
           INFN Sezione di Padova, Universit\`a di Padova, Padova, Italy \and
           SUPA - School of Physics and Astronomy, University of Glasgow, Glasgow, United Kingdom \and
           STFC Rutherford Appleton Laboratory, Didcot, United Kingdom \and
           Laboratoire Leprince-Ringuet, Ecole Polytechnique, CNRS/IN2P3, Universit\'e Paris-Saclay, Palaiseau, France \and
           Lunds Universitet, Fysiska Institutionen, Avdelningen f\"or Experimentell H\"ogenergifysik, Box 118, 221 00 Lund, Sweden \and
           University of California, San Diego, La Jolla, CA, USA
          }

\abstract{The increase in the scale of LHC computing expected for Run
  3 and even more so for Run 4 (HL-LHC) over the next ten years will
  certainly require radical changes to the computing models and the
  data processing of the LHC experiments. Translating the requirements
  of the physics programmes into computing resource needs is a
  complicated process and subject to significant uncertainties. For
  this reason, WLCG has established a working group to develop
  methodologies and tools intended to characterise the LHC workloads,
  better understand their interaction with the computing
  infrastructure, calculate their cost in terms of resources and
  expenditure and assist experiments, sites and the WLCG project in
  the evaluation of their future choices.  This working group started
  in November 2017 and has about 30 active participants representing
  experiments and sites. In this contribution we expose the
  activities, the results achieved and the future directions.}

\maketitle

%Sciaba
\section{Introduction}
The computing infrastructure for the LHC experiments, managed via the
WLCG project~\cite{wlcg}, has been successfully operating since 2008,
with a good match between the resources needed by the experiments and
those made available by the funding bodies. In the last few years
though it became increasingly clear that Run 3 (for ALICE and LHCb)
and Run 4, or HL-LHC (for ATLAS and CMS) will determine very
significant increases in the scale of computing, which will not be
simply accommodated by technological improvements in the likely
scenario of a flat budget evolution. Simply extrapolating the current
software performance to the expected trigger rates and average pile-up
produces a $O(10)$ discrepancy between the needed and the affordable
levels of computing and storage capacity. For this reason,
``revolutionary'' changes in the software and the computing models of
the experiments will be absolutely necessary.

The need of a joint working group between WLCG and the
HSF~\cite{hsf} dedicated to the study of performance and cost of
computing became apparent and it finally started at the end of
2017, with a long term roadmap that extends to the start of
HL-LHC. About thirty members, from experiments, sites and IT and software
experts participate in the group activities. The initial focus has
been on improving the understanding of current workloads and to
establish methodologies and tools to analyse their performance; however,
some thought has already been given to the exploration of future
scenarios and to try to quantify the gains that could be achieved
through paradigm changes. Currently the most important areas of work are:
\begin{itemize}
\item the collection of reference workloads from each experiment, to conduct
  performance studies in controlled and repeateable conditions;
\item the definition of the metrics that best characterise the applications and the implementation of tools to measure them;
\item the adoption of a common framework for estimating resource needs;
\item the adoption of a common process to evaluate the cost of an infrastructure
  as a function of the experiment needs;
\item preliminary studies to explore possible savings in different areas.
\end{itemize}



%Markus
\section{Workload characterisation and metrics}
\begin{figure}[h]
  \centering
  \includegraphics[height=6cm]{CHEP-Model-2.png}
  \caption{Workload resource requirements of the workloads depend on
    the pileup and the trigger rate.  With this information the
    model predicts the demands on the different resources. The Fabric
    Mapper abstracts the characteristics of a fabric and produces
    estimates for throughputs. Optional sites can add their local cost
    structure to the model}
  \label{fig:mapping}
\end{figure}

To model the behaviour of the workloads of the LHC experiments it is
essential to understand how the different capabilities that a site
provides impact the throughput of a specific workload. To limit the
complexity of a performance model, it is desirable to minimise the
number of performance characteristics; this requires to find a set of
``orthogonal'' metrics to which the workload throughput is
sensitive. This knowledge can then be used by software developers to
avoid bottlenecks by balancing resource usage, and by site managers to
optimise their expenditure. The balance between the amount of memory
per core, the disk performance and the speed of the memory can be
optimised on the basis of the characterisation of the relevant
workloads. Figure~\ref{fig:mapping} illustrates this relationship.

\begin{figure}[h]
  \centering
  \includegraphics[height=6cm]{prmon.png}
  \caption{Memory consumption over the execution of an ATLAS Digi-Reco
    job; the different processing steps are clearly visible. The periods
    with very low memory consumption are due to the merging of
    intermediate files}
  \label{fig:prmon}
\end{figure}

Finding such a metric set if far from trivial, given the complexity of
workloads and their production environments. We started from a
representative collection of reference workloads for each experiment.
Then, we listed several metrics, grouped in various categories (CPU,
I/O, memory, storage, swap and network) and assessed their
relevance. Finally we focused on those that can be measured without
significant interference with the running workload on nodes. A tool,
PrMon \cite{prmon}, originally developed by ATLAS, has been
generalised and allows to measure time series of a large set of
parameters. Prmon uses information collected by the operating system
and does not require any instrumentation. Figure~\ref{fig:prmon}
shows, as an example, measurements of the memory usage during the
execution of a workload that performs digitisation with pile-up data
and reconstruction for Monte Carlo events.  Prmon allows to estimate
which capabilities of a system limit the throughput and how
efficiently multi-process and multi-threaded version of the programs
use their allocated cores. To study the dependencies on latency,
bandwidth and memory restrictions, another tool has been developed
that runs workloads repeatedly with increasingly restricted bandwidth,
memory and increased latency, measuring the throughput for each
configuration.

Another tool, Trident~\cite{trident}, was developed to provide a
convenient access to the information contained in CPU hardware
counters.  These counters provide information on the level of
parallelism exploited by a workload, memory access, utilisation of
caches and vector capabilities, etc.~in an abstract and CPU-model
dependent way.  This gives the developer an estimate on how much
improvement is at best possible in exploiting the available resources
and quantitatively understand the limiting factors. The site managers
can use the insights gained from Trident in decisions concerning
trade-offs between different capabilities, like faster memory vs. size
or system disk speed and network.

\subsection{Next steps}
While the time series measured by Prmon give valuable insights, they
cannot be directly used as input for modeling the behaviour of the
workloads. Work on this process has started to extract parameters from
metric time series, as processing can be described as sequence of
steps, each one looping over events. During each of these steps the
resource usage can be described by a small number of parameters and
the number of processed events. This, in combination with the measured
dependency on latency, bandwidth and memory, will allow to predict the
throughput of the current workloads under different conditions.

For the resource modelling under future running conditions the
dependency on the average pile-up has to integrated into the
model. For data size and simulation time, the dependency is at most
linear. For reconstruction, an extrapolation to HL-LHC levels is
currently very uncertain and it would be exponential with the current
algorithms; ATLAS and CMS planned detector upgrades will help to
address this problem. ATLAS by introducing tracking layers that are
closely spaced, so that they provide precise starting vectors and CMS
by high precision time resolution that allows to distinguish between
hits from different collisions.


%Sartirana
\section{Resource estimation}
Explain why we need a framework to calculate resource estimates.

Describe the common framework derived by Ken Bloom.

Describe the work needed to be utilised by other experiments.

$\leq 1$ page.


%Renaud
\section{Site cost estimation}
%Explain the importance of a common method to estimate the costs for sites
%given the resource needs of the experiments.
%Describe Renaud's model and future perspectives for this area.

\subsection{Context}
%The IT resources deployed in the data centres contributing to WLCG
%are an important source of expense for funding agencies.  With the
%expected amount of data to be recorded at HL-LHC, it is highly
%desirable for WLCG to estimate the amount of IT resources that will
%be available for data processing in sites over time.  Simple
%projections from the current trends in the IT market, assuming a
%constant budget in sites over time, indicate that the available
%amount of resources across the WLCG will not be sufficient at all to
%process the HL-LHC data (reference?).  Therefore, it becomes clear
%that, unless technology revolutions happen in the meantime, LHC
%experiments will need to make a different usage of IT resources that
%sites will deploy.

The purpose of site cost estimation is to understand and measure, at a
global scale, what the data centres typical expenses are, and predict
what they may be in years from now.  The approach chosen in this
context is to model those expenses, taking into account the diversity
of national contexts across sites in terms of funding, procurement
procedures, and local market conditions.  These results will help
experiments plan new computing strategies that will improve the
cost-effectiveness of their resource usage.

\subsection{First results}

Some of the first elements to address are the diversity of expenses
across sites, and the definition of what these expenses are.  A simple
and quick exercise was made by four candidate sites, aiming to get a
first estimate of the financial cost to run a given workflow and to
store a given amount of data, in terms of IT resources and power
consumption.  The answers to this exercise happened to be
significantly different from site to site, up to a factor of 2.  The
reasons were found to originate from different aspects: the intrinsic
variety of costs, the measurement method, and the understanding of
what a given metric means.

Those results showed clearly the need of a consolidated and common
model and method to measure costs across WLCG data centres.  Here are
a few examples to illustrate this point. One may consider the cost of
a server providing a given capacity, including (or not) the equipement
that is shipped with it (rack, switch, adaptor etc.).  One may also
consider that one capacity unit of tape storage includes (or not) the
investments made in library, drives, disk cache etc.  that are needed
for such system to work properly.  Finally, the measurement of an
electrical consumption may include (or not) the Power Usage
Effectiveness of the data centre in order to take into account UPS or
HVAC system contributions to the final power bill.

Consequently, it becomes fundamental in the context of site cost
estimation to establish precise definition of the cost-related metrics
in order to build a reliable model.

\subsection{Next steps}

A preliminary attempt to address data centre resource TCO through cost
modeling was shown in reference~\cite{costmodel}.  This model assumes
that a data centre invests year after year a constant budget in the
following assets: batch system, disk storage and tape system
capacities.  If this hypothesis is satisfied (even roughly), one can
show that budget ($B$) and available capacity ($K$) over time are
bound together by a quantity ($c^*$) that depends on hardware cost
evolution and lifetime:

\begin{equation}
    B (t) = K (t) \times c^* (t)
    \label{eq:costmodel}
\end{equation}

In the case where hardware costs by unit of capacity decrease
exponentially over time with a rate $r$ (e.g. 0.2 for a 20\% yearly
decrease) and hardware is replaced after $\tau$ years, one can show
that

\begin{equation}
c^*(t)=c(t)\frac{r}{1-(1-r)^\tau}
\end{equation}
and the time evolution of the site capacity for a flat budget turns
out to be exponentially increasing.

This model however does not address all the components of a TCO, like
manpower.  Site cost studies may leverage this model to estimate the
financial impact of a variation in the usage that experiments make of
data centre IT resources.  But the quantity $c^*$ may differ
significantly from site to site, so an extension of this model at
global scale will have to take into account as many as site-dependent
parameters as possible to establish $c^*$.

In order to get a better idea of the variations of the expenditure at
different sites, a survey is being conducted at all Tier-1 sites,
although the results are not yet available.


%Markus
\section{HL-LHC and areas of improvement}
Since the start of the LHC the community constantly improved the
throughput of the main workflows; however, studies conducted by the
Understanding Performance Team and the Cost and this working group
still found several areas in which potential improvements could be
achieved~\cite{improvements}.

\subsection{Compiler and software improvements}
WLCG sites provide a variety of CPU models, and almost all code is
built to run on the oldest available architectures in the system. This
and recent advancements in compiler techniques motivated further
studies.

From studies conducted on current GEANT simulation, reconstruction and
NLO generator code, gains by compiler and link-based optimisations are
around 20-25\%. This includes the compilation of the code for
individual target CPUs, the use of Intel’s commercial compiler and the
use of feedback-directed optimisation (AutoFDO from Google and
Intel). Attempts of compiler-based vectorisation of current production
codes showed none or minimal improvements. Reducing the overhead of
shared libraries by building large libraries resulted in significant
gains on older architectures (e.g., a 10\% improvement on Ivy Bridge
for the ATLAS simulation code).  Comparisons between the Intel
compiler and current gcc versions showed no clear differences. It has
to be noted that the use of feedback-directed optimisation requires
that the objects are build statically, which for some of the LHC
codebase is not trivial.

From profiling the code and detailed analysis of the dynamic use of
memory allocation, it is known that the current code spends up to 25\%
of the time on memory/object management, due to the frequent creation
and destruction of small objects~\cite{fomtools}. A 60-90\% of
allocations exists for less than $100\mu$s and are smaller than 64
Bytes.  By using more adequate techniques and strategies for object
management, such as object pools, libs for large and static vectors,
this could be reduced to less than 10\% with relative minor code
refactorisation. At the same time the data structure layout can be
improved for more efficient utilisation of caches and improved memory
access.

The current HEP code executes on modern cores significantly less than
2 instructions per cycle (0.8-1.5) . With the large vector registers
provided by current cores the theoretical limit can be above 20
instructions/cycle for code that can be completely
vectorised. However, tuned complex code for HPC systems can reach
values of about 4. This can be seen as an upper limit for our code
base. Approaching this level of utilisation requires at least a change
of the used data structures and a re-implementation/factorisation of
the algorithms used in the HEP code base. This is therefore best taken
into consideration when new algorithms are designed and implemented.

Gains from new algorithms and the impact of new detector components
can’t be treated here. As the development of the new, cellular
automaton based, HLT track reconstruction code for the ALICE HLT has
shown, large factors O(100) can in some cases be achieved~\cite{rohr}.

\subsection{Storage}
The LHC community has recently agreed to a scenario where managed
storage is consolidated at a few, very large sites (the ``data lake'')
as the most promising to achieve significant cost savings.

For what concerns operational effort, the 2015 WLCG site
survey~\cite{survey} showed that on average Tier-1 sites require 2.5
FTEs for operating storage and Tier-2 sotes 0.75 FTEs, with a weak
dependence on the amount of storage (+15\% FTEs for doubling the
storage). By concentrating managed storage at a few sites and using
only disk caches (much simpler to manage) at most sites, one can
estimate a decrease of the overall number of FTE for storage
operations from around 100 to around 60, that is, a 40\% reduction.

The vast majority of disk ($~80\%$) is used for data formats used for
analysis. Data popularity studies show that on average datasets exist
at about 2 sites and are accessed less than 10 times over a period of
six months, with most accesses happening in the first month.  We can
argue that data needs to be stored only once at the large storage
sites, and just be temporarily cached as needed depending on the
client load. Less popular data may be completely purged from disk and
kept on tape.

How much saving can be expected depends on the retention strategy and
the use of hierarchical storage (with tape costing about 1/4 than disk
storage). The savings from replacing some amount of disk with tape are
partially offset by the need of more tape drives and added complexity
in the data migration (the latter assumed not to be substantial due to
the sophistication of the current data management systems).

Again, analysis of popularity data and storage system monitoring
indicates that only a small fraction of the produced data is active at
any time and a significant fraction (15-20\%) of the data could be
moved to a different storage layer.

The concentration of data at a few sites requires to limit the impact
of bandwidth limitations and latency on the throughput of
applications.  We conducted measurements of the impact of latency to
the throughput of various workloads leading to the understanding that
for latencies up to 25ms the reduction of throughput can be limited to
5\% when using Xcache as an additional layer. Some workloads already
manage latency on the client level very well. The results are
summarised in table~\ref{tab:latency}.
\begin{table}
  \centering
  \caption{ Summary of Cache and Latency Studies}
  \label{tab:latency}
  \begin{tabular}{lllll}
    \hline
    \textbf{Workload} & \textbf{Condition} & \textbf{Latency (ms)} & \textbf{Method} & \textbf{Rel. time}  \\\hline
    ATLAS Digi-Reco & data on node & 0 & running local & 1 \\ 
    ATLAS Digi-Reco & data remote & ~25 & running local & 1.9 \\ 
    ATLAS Digi-Reco & data remote, empty cache & ~25 & Xcache & 1.08 \\ 
    ATLAS Digi-Reco & data remote, pop. cache & ~25 & Xcache & 1.04 \\ 
    ATLAS Derivation & data on node & 0 & running local & 1 \\ 
    ATLAS Derivation & data on EOS & < 1 & running local  & 1.02 \\ 
    ATLAS Derivation & data remote & ~25 & running local & 8.3 \\ 
    ATLAS Derivation & data remote, empty cache & ~25 & Xcache & 1.05 \\ 
    ATLAS Derivation & data remote, pop. cache & ~25 & Xcache & 1.03 \\
    CMS Digi-Reco & data local & 0 & added latency & 1 \\
    CMS Digi-Reco & data local & 5 & added latency & 1.01 \\
    CMS Digi-Reco & data local & 10 & added latency & 1.04 \\
    CMS Digi-Reco & data local & 20 & added latency & 1.11 \\
    CMS Digi-Reco & data local & 50 & added latency & 1.24 \\\hline
  \end{tabular}
\end{table}

Data is also replicated within storage systems to ensure a high degree
of reliability, either by replication (more performant but expensive)
or some form of error encoding (cheaper but less performant). Even
larger savings can be achieved by not using storage redundancy at all
and re-staging from tape, or possibly even regenerating, any lost
data.

Based on the observed disk failure rate of an individual disk in the
CERN EOS system, which is about 1\% per year, the relative total cost
of storage and computing at CERN (around 4 HS06/TB), the amount of CPU
time to generate AOD events ($\sim 850$ HS06$\times s$) and their size
($\sim 400$ kB), one can naively estimate that the computing cost to
re-generate the AOD data lost to disk failures is $\sim 20\%$ of the
cost to make the storage that contained it fully redundant.

In reality, most of the times the lost data would be replicated
elsewhere, although it would not be the case for intermediate data
sets; one can conservatively estimate that 30-50\% of the disc costs
can be saved. For this approach to be efficient and effective the
process of recreating individual files has to be automated, which is
highly desirable since other failure modes lead to data loss on a
comparable scale.

\subsection{Gains from improvements in operations}
Scheduling inefficiencies in WLCG arise from many reasons. Either due
to a mismatch between cores in a system and memory requirements,
mismatch between requested cores and cores grouped on nodes (the
tessellation problem), batch system inefficiencies, pilot service
inefficiencies, delays due to data staging, I/O waits etc.. These
inefficiencies are different for different sites and workloads. Site
managers have identified some of these problems and estimates range
from 20\% - 30\% of resources being lost due to one or another
scheduling inefficiency. It has been shown that with advanced
backfilling and more complex job placement strategies, efficiencies
above 90\% can be reached, at the expense of higher complexity of site
and experiment workload management systems. To efficiently use short
usage windows the granularity of workloads has to be
increased. Another source of scheduling inefficiencies stems from
breaking the processing chains from raw data to data analysis objects
into several individual steps that exchange data via
files. Experiments have started to chain these steps, but often still
rely on intermediate local files. Especially when parallel processing
threads/processes write individual files, the merging steps create
inefficiencies, as they are single threaded. By using shared writers
these inefficiencies can be reduced. The overall impact is difficult
to measure, but from ATLAS pileup/digitisation/reconstruction chains
activity logs it can be estimated to be about 5\%.

Losses due to intermittent job failures are for complex workflows
unavoidable. Currently most of the steps can recover from failure with
the help of automated retry and failover mechanisms. Nevertheless
complete jobs are sometimes run up to 20 times before successful
completion. The overall loss in walltime due to job failures is
between 10 and 15\%. Improved procedures to make jobs more resilience
or fail very early can reduce these losses, but will not eliminate
them.

Table~\ref{tab:pgain} summarises the different identified potential gains.

\begin{table}
  \centering
  \caption{Estimated potential gains and associate efforts}
  \label{tab:pgain}
  \begin{tabular}{llll}
    \hline
    \textbf{Change} & \textbf{Effort: Sites} & \textbf{Effort: Users} & \textbf{Potential Gain}  \\\hline
    Managed storage only at sites & some on large sites & little & -40\% ops effort \\
    Reduced data redundancy & some on larger sites & some & -30-50\% disk cost \\
    Scheduling and site inefficiencies & some & some & +10-20\% CPU  \\
    Reduced job failure rates & little & some - massive & +5-10\% CPU \\
    Compiler and build improvements & none &  little - some & +15-20\% CPU \\
    Improved memory usage & none & some &  +10-15\% CPU \\ 
    Exploiting modern CPU arch. &  none & massive & +100\% CPU \\\hline
  \end{tabular}
\end{table}


%Sciaba
\section{Conclusions}
Conclusions.

$\leq 0.5$ pages.


\begin{thebibliography}{}
\bibitem{wlcg}
Worldwide LHC Computing Grid, http://wlcg.web.cern.ch/
\bibitem{hsf}
HEP Software Foundation, https://hepsoftwarefoundation.org/
\bibitem{prmon}
G.~Stewart, A.S.~Mete, https://github.com/HSF/prmon
\bibitem{trident}
S. Muralidharan, D. Smith, \textit{Trident: A three pronged approach to analysing node utilisation}, these proceedings
\bibitem{fomtools}
S.~Kama and N.~Rauschmayr, J.\ Phys.: Conf.\ Ser.\ {\bf 898} no.7, 072031 (2017)
\bibitem{cwp}
HEP Software Foundation, \textit{A Roadmap for HEP Software and Computing R\&D for the 2020s}, arXiv:1712.06982
\bibitem{survey}
M. Alandes Pradillo {\em et al}, J. Phys.: Conf. Ser. \textbf{664} 032025 (2015)
\bibitem{ourresmodel}
A.~Sartirana, https://github.com/sartiran/resource-modeling/tree/wlcg-wg-new
\bibitem{cmsresmodel}
D.~Lange {\em et al}, \textit{Computing Resources: Meeting the demands of the high-luminosity LHC physics program}, these proceedings
\bibitem{costmodel}
R.~Vernet, J.\ Phys.: Conf.\ Ser. \textbf{664}, 052040 (2015)
\end{thebibliography}


\end{document}
