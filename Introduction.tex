\section{Introduction}
The computing infrastructure for the LHC experiments, managed via the
WLCG project, has been successfully operating since 2008, with a good
match between the resources needed by the experiments and those made
available by the funding bodies. In the last few years though it
became increasingly clear that Run 3 (for ALICE and LHCb) and Run 4,
or HL-LHC (for ATLAS and CMS) will determine very significant
increases in the scale of computing, which will not be accommodated by
technological improvements in the likely scenario of a flat budget
evolution. Simply extrapolating the current software performance to
the expected trigger rates and average pileup produces an O(10)
discrepancy between the needed and the affordable levels of computing
and storage capacity. For this reason, ``revolutionary'' changes in
the software and the computing models of the experiments will be
absolutely necessary.

Consequently the need of a joint HSF/WLCG working group dedicated to
the study of performance and cost of computing became obvious and it
was finally started at the end of 2017, with a long term roadmap that
extends to the start of HL-LHC. About 30 members, from experiments,
sites and IT and software experts participate to the group
activities. The initial focus has been on improving the understanding
of current workloads and to establish methodologies and tools to
analyse performance; however, some thought is being already given to
the exploration of future scenarios and to try to quantify the gains
that could be achieved through paradigm changes. The most important
areas of work are the following:
\begin{itemize}
\item collection of reference workloads from each experiment, to conduct
  performance studies in controlled and repeateable conditions;
\item definition of the metrics that best characterise the applications and implementation of tools to measure them;
\item adoption of a common framework for estimating resource needs:
\item adoption of a common process to evaluate the cost of an infrastructure
  as a function of the experiment needs.
\end{itemize}

